\documentclass[]{article}
\usepackage{lmodern}
\usepackage{amssymb,amsmath}
\usepackage{ifxetex,ifluatex}
\usepackage{fixltx2e} % provides \textsubscript
\ifnum 0\ifxetex 1\fi\ifluatex 1\fi=0 % if pdftex
  \usepackage[T1]{fontenc}
  \usepackage[utf8]{inputenc}
\else % if luatex or xelatex
  \ifxetex
    \usepackage{mathspec}
  \else
    \usepackage{fontspec}
  \fi
  \defaultfontfeatures{Ligatures=TeX,Scale=MatchLowercase}
\fi
% use upquote if available, for straight quotes in verbatim environments
\IfFileExists{upquote.sty}{\usepackage{upquote}}{}
% use microtype if available
\IfFileExists{microtype.sty}{%
\usepackage{microtype}
\UseMicrotypeSet[protrusion]{basicmath} % disable protrusion for tt fonts
}{}
\usepackage[margin=1in]{geometry}
\usepackage{hyperref}
\hypersetup{unicode=true,
            pdftitle={Exploring probability distributions for bivariate temporal granularities},
            pdfauthor={Sayani Gupta -- Department of Econometrics and Business Statistics, Monash University},
            pdfborder={0 0 0},
            breaklinks=true}
\urlstyle{same}  % don't use monospace font for urls
\usepackage{graphicx,grffile}
\makeatletter
\def\maxwidth{\ifdim\Gin@nat@width>\linewidth\linewidth\else\Gin@nat@width\fi}
\def\maxheight{\ifdim\Gin@nat@height>\textheight\textheight\else\Gin@nat@height\fi}
\makeatother
% Scale images if necessary, so that they will not overflow the page
% margins by default, and it is still possible to overwrite the defaults
% using explicit options in \includegraphics[width, height, ...]{}
\setkeys{Gin}{width=\maxwidth,height=\maxheight,keepaspectratio}
\IfFileExists{parskip.sty}{%
\usepackage{parskip}
}{% else
\setlength{\parindent}{0pt}
\setlength{\parskip}{6pt plus 2pt minus 1pt}
}
\setlength{\emergencystretch}{3em}  % prevent overfull lines
\providecommand{\tightlist}{%
  \setlength{\itemsep}{0pt}\setlength{\parskip}{0pt}}
\setcounter{secnumdepth}{0}
% Redefines (sub)paragraphs to behave more like sections
\ifx\paragraph\undefined\else
\let\oldparagraph\paragraph
\renewcommand{\paragraph}[1]{\oldparagraph{#1}\mbox{}}
\fi
\ifx\subparagraph\undefined\else
\let\oldsubparagraph\subparagraph
\renewcommand{\subparagraph}[1]{\oldsubparagraph{#1}\mbox{}}
\fi

%%% Use protect on footnotes to avoid problems with footnotes in titles
\let\rmarkdownfootnote\footnote%
\def\footnote{\protect\rmarkdownfootnote}

%%% Change title format to be more compact
\usepackage{titling}

% Create subtitle command for use in maketitle
\providecommand{\subtitle}[1]{
  \posttitle{
    \begin{center}\large#1\end{center}
    }
}

\setlength{\droptitle}{-2em}

  \title{Exploring probability distributions for bivariate temporal granularities}
    \pretitle{\vspace{\droptitle}\centering\huge}
  \posttitle{\par}
    \author{Sayani Gupta -- Department of Econometrics and Business Statistics,
Monash University}
    \preauthor{\centering\large\emph}
  \postauthor{\par}
    \date{}
    \predate{}\postdate{}
  
\usepackage{todonotes,mathpazo}

\begin{document}
\maketitle

Smart meters measure energy usage at fine temporal scales, and are now
installed in many households around the world. We propose some new tools
to explore this type of data, which deconstruct time in many different
ways. There are several classes of time deconstructions including linear
time granularities, circular time granularities and aperiodic calendar
categorizations. Linear time granularities respect the linear
progression of time such as hours, days, weeks and months. Circular time
granularities accommodate periodicities in time such as hour of the day,
and day of the week. Aperiodic calendar categorizations are neither
linear nor circular, such as day of the month or public holidays.

The hierarchical structure of many granularities creates a natural
nested ordering. For example, hours are nested within days, days within
weeks, weeks within months, and so on. We refer to granularities which
are nested within multiple levels as ``multiple-order-up''
granularities. For example, hour of the week and second of the hour are
both multiple-order-up, while hour of the day and second of the minute
are single-order-up.

Visualizing data across various granularities helps us to understand
periodicities, pattern and anomalies in the data. Because of the large
volume of data available, using displays of probability distributions
conditional on one or more granularities is a potentially useful
approach. This work provides tools for creating granularities and
exploring the associated within the tidy workflow, so that probability
distributions can be examined using the range of graphics available in
the \href{https://cran.r-project.org/package=ggplot2}{ggplot2} package.
In particular, this work provides the following tools.

\begin{itemize}
\item
  Functions to create multiple-order-up time granularities. This is an
  extension to the
  \href{\%5Bhttps://cran.r-project.org/package=lubridate}{lubridate}
  package, which allows for the creation of some calendar
  categorizations, usually single-order-up.
\item
  Checks on the feasibility of creating plots or drawing inferences from
  two granularities together. Pairs of granularities can be categorized
  as either a \emph{harmony} or \emph{clash}, where harmonies are pairs
  of granularities that aid exploratory data analysis, and clashes are
  pairs that are incompatible with each other for exploratory analysis.
\end{itemize}


\end{document}
